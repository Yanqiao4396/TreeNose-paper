\section{Background and Related Work}
\label{sec:background}

Code smell origins from the book \textit{Refactoring} by Martin Fowler and Kent Beck \cite{Fowler_Beck}.
Since then, this concept has been widely adopted by the software engineering community. The community has
defined more specific code smells \cite{Pysmell} \cite{CleanCode} \cite{SQLAntipatterns} 
\cite{RefactoringWorkbook} to describe the issues in software development. Jerzyk, recently, built
an online code smell catalog with 56 common code smells \cite{Jerzyk2023}.
At the same time, community was also exploring the characteristics of code smells. Yamashita 
conducted several empirical studies to investigate the effects of code smells on maintainability \cite{6405287} \cite{6392174}. She
also conducted another study to discover developers' opinion of code smells \cite{developersCare}, where 
she found the top 15 most common code smells in developers' perception. Researchers, like Tufano \cite{whenandwhy} and Peters \cite{lifespan}, conducted the studies to
discover the characteristics of code smells in lifespan of software projects. Santana, Cruz, and, Figueiredo found the
strong agglomerations of certain code smells highly occur in software projects \cite{Santana}. Pascarella found active code reviews significantly reduce the severity of code smells \cite{Pascarella}.


Because of the negative impact of code smells on software projects, developers acknowledge the importance of detecting them. The manual code smell detection was proven to be
error-prone and resource-consuming \cite{DetectingDefectsInObject-orientedDesigns}. Therefore, code smell detection tools have been developed to automate this process.
Different code smell detection tools rely on various detection strategies, such as abstract syntax tree (AST) analysis, Machine Learning, and Static Analysis.
Emden and Moonen built the first code smell detection tool in Java \cite{1173068}. Since then,
many language-specific code smell detection tools have been developed, such as PMD \cite{PMD} and CheckStyle \cite{CheckStyle}.
Chen \cite{Pysmell} built a code smell detection tool in Python targeting on Python specific code smells.
All the tools mentioned above rely on AST to detect code smells. There are also developers who use Machine Learning. 
For example, Pontillo utilized Machine Learning to detect test smells, similar to code smell, to achieve better performance than heuristic-based techniques, but had a challenge of overcoming F1 score of 0.51.

