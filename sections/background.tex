\vspace*{-0.5em}

\section{Background and Related Work}~\label{sec:background}

\vspace*{-1em}

% Explain the basics of code smells and their importance and prevalence

{\bf Code Smells}. Originating from Fowler and Beck's \textit{Refactoring}
book, code smells are important considerations for both researchers and
practitioners. The research community has defined numerous code smells
\cite{Pysmell,SQLAntipatterns,CleanCode,RefactoringWorkbook}, with Jerzyk
recently releasing an online catalog with 56 common code smells
\cite{Jerzyk2023}. At the same time, Yamashita investigated the effects of code
smells on maintainability (e.g., \cite{6392174, 6405287}) and studied
developers' opinion of code smells \cite{developersCare}, ultimately finding
what real-world developers consider the top 15 most common code smells.
Moreover, Tufano~\etal{} \cite{whenandwhy} and Peters and
Zaidman~\cite{lifespan} characterized the lifespan of code smells in software
projects. Santana, Cruz, and, Figueiredo found the strong agglomerations of
certain code smells highly occur in software projects \cite{Santana}.
Pascarella found active code reviews significantly reduce the severity of code
smells \cite{Pascarella}.

{\bf Code Smell Detection}. Because of the negative impact of code smells on
software projects, developers acknowledge the importance of detecting them. The
manual code smell detection was proven to be error-prone and resource-consuming
\cite{DetectingDefectsInObject}. Therefore, code smell detection tools have been
developed to automate this process. Different code smell detection tools rely on
various detection strategies, such as abstract syntax tree (AST) analysis,
Machine Learning, and Static Analysis. Van Emden and Moonen built the first code
smell detection tool in Java \cite{1173068}. Since then, many language-specific
code smell detection tools have been developed, such as PMD \cite{PMD} and
CheckStyle \cite{CheckStyle}. Chen \cite{Pysmell} built a code smell detection
tool in Python targeting on Python specific code smells. All the tools mentioned
above rely on AST to detect code smells. Except AST-based detection, There are
also developers utilizing approaches like Machine Learning and textual
components. For example, Pontillo utilized Machine Learning to detect test
smells, similar to code smell, to achieve better performance than
heuristic-based techniques, but had a challenge of overcoming F1 score of 0.51.
Palomba~\etal{} designed a text-based code smell technique to detect code smells
and found the benefits of combining textual and structural information in code
smell detection \cite{Palomba}.

% TODO: Reference Saavedra 2023 paper; it uses an intermediate representation

{\bf Multi-Language Code Smells}. It's one feature of modern software
development to use multiple programming languages~\cite{723183}. Developers
utilize multiple programming languages to leverage their
strengths~\cite{7476675}.
%
Common code smells are broadly recognized across programming languages
\cite{PMD,CheckStyle,Pysmell,Jscent,DesigniteJava}. There are also researchers
who have conducted studies to investigate code smells in multi-language
systems. Abidi~\etal{} defined 12 multi-language design smells (i.e., code
smells and anti-patterns) \cite{MultiLanguageCodeSmells}. They discovered that
multi-language code smells commonly raise during the maintenance and
refactoring activities and have a strong negative impact on software quality
\cite{Abidi2}. They also built a multi-system code smell detection tool with
srcML (a multi-language parsing tool to convert source code into XML), to
detect design smells in multi-language systems. They suggested that
multi-language design smells are prevalent in multi-language systems and have a
strong negative impact on software readability \cite{Fault-Prone}. Nagy and
Cleve built a SQL code smell detection tool for SQL statements embedded in Java
code \cite{SQLInJava}.
