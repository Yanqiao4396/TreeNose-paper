\vspace*{-1em}

\section{Background and Related Work}~\label{sec:background}

\vspace*{-0.5em}

% Explain the basics of code smells and their importance and prevalence

{\bf Code Smells}. Originating from Fowler and Beck's \textit{Refactoring}
book, code smells are important considerations for both researchers and
practitioners. The research community has defined numerous code
smells~\cite{Pysmell,SQLAntipatterns,CleanCode,RefactoringWorkbook}, with
Jerzyk recently releasing an online catalog with 56 common code
smells~\cite{Jerzyk2023}. At the same time, Yamashita investigated the effects
of code smells on maintainability (e.g.,~\cite{6392174, 6405287}) and studied
developers' opinion of code smells~\cite{developersCare}, ultimately
identifying what real-world developers consider the top 15 most common code
smells. Moreover, Tufano~\etal{}~\cite{whenandwhy} and Peters and
Zaidman~\cite{lifespan} characterized the lifespan of code smells in software
projects. Finally, Santana~\etal{} found that strong agglomerations of certain
code smells often occur in software projects~\cite{Santana} and
Pascarella~\etal{} showed that code reviews significantly reduce the severity
of code smells~\cite{Pascarella}.

% Cut content:

% Therefore, code smell detection tools have been developed to automate this
% process. Different code smell detection tools rely on various detection
% strategies, such as abstract syntax tree (AST) analysis, Machine Learning, and
% Static Analysis. Van Emden and Moonen built the first code smell detection
% tool in Java \cite{1173068}.

% Provide a very brief overview of code smell detection tools.
% Key insight to get across:
% - AST
% - Text
% - Machine Learning
% Real tools called linters

{\bf Code Smell Detection}. Due to their negative influence on software
quality, developers acknowledge the importance of detecting code
smells~\cite{developersCare} --- even though manual code smell detection is
both error-prone and resource-consuming~\cite{DetectingDefectsInObject}.
%
To automate this process, different code smell detection tools analyze either a
program's abstract syntax tree (AST)~\cite{Lenarduzzi2023} or its
text~\cite{Palomba}, or use machine learning~\cite{ML}.
%
Open-source developers have also created many language-specific linters for
smell detection (e.g., PMD~\cite{PMD} and CheckStyle~\cite{CheckStyle}), while
researchers created and evaluated ones like PySmell~\cite{Pysmell}.

% Cut content:

% built a code smell detection tool in Python targeting on
% Python specific code smells.

% All the tools mentioned
% above rely on AST to detect code smells. Except AST-based detection, there are
% also developers utilizing approaches like Machine Learning and textual
% components. For example, Pontillo utilized Machine Learning to detect test
% smells, similar to code smell, to achieve better performance than
% heuristic-based techniques, but had a challenge of overcoming F1 score of 0.51.
% Palomba~\etal{} designed a text-based code smell technique to detect code smells
% and found the benefits of combining textual and structural information in code
% smell detection \cite{Palomba}.

% Since modern software systems use
% ~\cite{723183}.

% It's one feature of modern software development to use multiple programming
% languages%

% TODO: Reference Saavedra 2023 paper; it uses an intermediate representation

{\bf Multi-Language Code Smells}.
%
Since developers mix and match the best features of programming
languages~\cite{7476675}, real-world software is often multiple-language in
nature~\cite{723183}.
%
Although many language-specific smell detectors find similar code
smells~\cite{PMD,CheckStyle,Pysmell,Jscent,DesigniteJava}, they do so with
differing internal representations and defaults, making their use difficult
either across many single-language projects or in any multi-language ones.
%
In response, smells in multi-language systems. Abidi~\etal{} defined 12
multi-language design smells (i.e., code smells and anti-patterns)
\cite{MultiLanguageCodeSmells}. They discovered that multi-language code smells
commonly raise during the maintenance and refactoring activities and have a
strong negative impact on software quality \cite{Abidi2}. They also built a
multi-system code smell detection tool with srcML (a multi-language parsing
tool to convert source code into XML), to detect design smells in
multi-language systems.

They suggested that multi-language design smells are prevalent in
multi-language systems and have a strong negative impact on software
readability \cite{Fault-Prone}.

Nagy and Cleve built a SQL code smell detection tool for SQL statements
embedded in Java code \cite{SQLInJava}.
