\begin{abstract}

    % A code smell is a code-design issue that can lead to decrease in code
    % quality, thereby hindering developers from reading or maintaining a
    % software program.
    %
    Code smell detection tools help developers by automatically checking for the
    existence of code anti-patterns that can compromise code quality.
    %
    % Code smell detection tools have the potential to be language-independent,
    % because common code smells occur in many different programming languages.
    % % A language-independent code smell detector can set unified detection
    % rules across programming languages, which is especially useful in
    % multi-language software development. % However, few code smell detection
    % tools support multiple programming languages.
    %
    Despite the prevalence of common code smells across various programming
    languages, there is a noticeable the need for such tools that consistently 
    perform code smell detection
    for both multiple- and single-language projects.
    %    Despite both the prevalence of common code smells across various programming
    % languages and the need for tools that consistently perform code smell detection
    % for both multiple- and single-language projects, there is a noticeable lack of
    % such tools that support multiple programming languages.

    Filling the aforementioned gap, this paper presents a tool, called
    \texttt{TreeNose}, that uses well-established Treesitter source code parsers to
    detect code smells across programming languages.
    The presented tool automatically finds 5 types of code smells across 3
    languages (i.e., Python, Java, and JavaScript) and its adoption of Treesitter
    parsers enables future extensions.
    This paper answers three research questions to explore \texttt{TreeNose}'s
    effectiveness and to characterize the manifestation of code smells in different
    languages.
    We applied \texttt{TreeNose} to 9 open-source projects implemented in the 3
    chosen languages, comparing its results
    % We applied \texttt{TreeNose} to 16 open-source projects implemented in the 3
    % chosen languages and the combinations of those languages, comparing its results
    to those arising from the use of traditional, language-specific code smell
    detectors in a manual annotation study.
    These results showed that \texttt{TreeNose}'s classifications lead to an
    F1-score of 0.94.
    The following experiments within 16 systems in the 3 selected languages and the combinations of them,
    revealed that the Complex Conditional smell comprises
    % The experiments also revealed that the Complex Conditional smell comprises
    between 32\% and 51\% of the detected \mbpx{code smells}.
    We also discovered that Long Method smells make up 38\% of all those detected
    for JavaScript programs and Long Class smells constitute 30\% of those found in
    Java programs.
    Finally, the results uncovered evidence of each programming language having
    code smells that are more common for it than the other languages.
    %
    % that are significantly more common than the languages, suggesting
    % a connection between code smells and programming languages.
    % % that
    % % they have strong tendencies to have specific code smells.

\end{abstract}
