\begin{abstract}

A code smell is a code-design-related issue that can lead to decrease
in code quality, thereby hindering developers from reading or maintaining
a software program. Code smell detection tools help developers by
automatically checking the existence of code smells. Code smell detection
tools have the potential to be language-independent, because common code
smells occur in many different programming languages.
A language-independent code smell detector can set unified detection rules
across programming languages, which is especially useful in multi-language
software development. However, few code smell detection tools support multiple
programming languages. To fill the aforementioned gap, this paper presents a
tool, called \texttt{TreeNose}, for detecting code smells across programming languages.
\texttt{TreeNose} can detect 5 types of code smells across multiple languages
(i.e., Python, Java, and JavaScript) and its extensible design enables it
to support additional languages. This paper answers three research questions
to explore the quality of \texttt{TreeNose} and structural patterns of code smells
in different languages. In order to answer those questions, we evaluated
\texttt{TreeNose} on 15 open-source projects implemented
in the three chosen programming languages, comparing
its results to those arising from the use of traditional language-specific
code smell detectors. One of his paper’s key results is that \texttt{TreeNose} has a
precision above 90\% for the chosen code smells and subject programs.
The experimental results also reveal that code smells are language-independent,
with varying prevalence in the studied programming languages.

\end{abstract}
