\begin{abstract}
    %
    Code smell detection tools help developers by automatically checking for the
    existence of code anti-patterns that may compromise software quality.
    %
    Despite the prevalence of common code smells across various programming
    languages, there is a noticeable need for language-independent tools that
    consistently perform code smell detection for both multiple- and
    single-language projects.
    %
    Filling the aforementioned gap, this paper presents a tool, called
    \texttt{TreeNose}, that leverages well-established Treesitter source code
    parsers to detect smells across programming languages.
    The presented tool automatically finds 5 types of code smells across 3
    languages (i.e., Python, Java, and JavaScript) and its adoption of Treesitter
    parsers ensures extensibility.
    This paper empirically answers three research questions to explore
    \texttt{TreeNose}'s effectiveness and to characterize the manifestation of code
    smells in different languages.
    Performing a manual annotation study, we used 9 open-source projects to compare
    \texttt{TreeNose} to language-specific code smell detectors, ultimately
    revealing that \texttt{TreeNose}'s classifications lead to an F1 score of 0.94.
    %
    Augmenting the 9 projects with 7 more, including 4 projects that are
    multi-language in nature, revealed that the Complex Conditional smell comprises
    between 32\% and 51\% of the detected code smells.
    %
    The results also showed that Long Method smells make up 38\% of all those
    detected for JavaScript programs and Long Class code smells constitute 30\% of
    those found in Java programs.
    Finally, the experiments uncovered evidence of each programming language having
    code smells that are more common for it than the other studied languages.
    %
\end{abstract}
