\section{Approach}
\label{sec:approach}

This chapter resolves 2 questions: 1. The list of code smells that \texttt{TreeNose} will detects, 
and 2. The approach that \texttt{TreeNose} will use to detect code smells across multiple programming languages.


\subsection{Definitions of Selected Code Smells}
\label{sec:Definitions of Selected Code Smells}

Code smells was first defined by Folwer and Beck back to 1999 in their book \textit{Refactoring}, 
where they defined 22 kinds of code smells \cite{Fowler_Beck}. Sine then, many researchers dedicated to define more 
code smells and to improve the existing ones. This paper will focus on the code smells that are top 15 most common in 
an empirical study conducted by Yamashita \& Moonen \cite{6405287}.
Among them, we selected 5 code smells that occur across multiple programming languages: 1. Complex Conditional, 2. Long Class,
3. Long Method, 4. Long Parameter List, and 5. Long Message Chain. The definitions of these code smells are as follows:

\textbf{Complex Conditional (CC)}: This smell occur when a conditional clause contains too many conditions, 
such as nested if-else statements, and enormous switch-case statements. This smell makes the code hard to read and maintain on logic level.

\textbf{Long Class (LC)}: a class handles too much work. Normally it occurs with too many fields. Or a class has too many lines because of 
the extremely long fields. 

\textbf{Long Method (LM)}: a method grows too long. Normally it occurs with too many lines of code. The longer a method is, the harder it is to understand the procedure.

\textbf{Long Message Chain (LMC)}: a long chain of object calls. This smell occurs when a method or an attribute calls another method or an attribute, and so on.

\textbf{Long Parameter List (LPL)}: a method or a function has too many parameters. When a method has too many parameters, it often indicates that the method is doing too much work.

\subsection{Detection Procedure}
\label{sec:Detection Procedure}

\texttt{TreeNose} adapts the AST-based approach to detect code smells. With customized detection thresholds,
\texttt{TreeNose} verifies the AST parsed from source code against the detection metrics. Table~\ref{tab:metrics-and-thresholds} shows the detection 
metrics with default thresholds for each code smell.
\texttt{TreeNose} executes 3 steps to detect target code smells. 1. Extract source code from the project, 2. Parse the source code to AST, 3. Analyze the AST to detect code smells.


\textbf{Step 1: Extract Source Code}: \texttt{TreeNose} extracts recursively extract source code matching with the target programming language from the project. During this procedure, \texttt{TreeNose} fetches
all target files unless the file or the path is in the ignore list.

\textbf{Step 2: Parse Source Code to AST}: \texttt{TreeNose} parses the source code to AST with TreeSitter \cite{treeSitter}. 
TreeSitter is a parser generator tool that generates ASTs in multiple programming languages. 
It decouple the parser from the language grammar, making it possible to parse the source code in 
multiple programming languages with the same parser. \texttt{TreeNose} uses the TreeSitter parser to parse the source code 
to AST and then generalize the language-specific AST into a common AST.

\textbf{Step 3: Analyze AST to Detect Code Smells}: \texttt{TreeNose} analyzes the AST to detect code smells. 
During this procedure, \texttt{TreeNose} traverses the AST and checks the detection metrics against the detection thresholds.
Finally, \texttt{TreeNose} generates reports with the list of code smells detected in the project.
% Always reference figures like you would reference sections as demonstrated in
% the sections/background.tex

\input{tables/metric_table}