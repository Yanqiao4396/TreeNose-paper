\section{Experimental Evaluation of TreeNose}~\label{sec:evaluation}

\vspace*{-1em}

% Integration of content to save space and be direct!
% 1) Describe the context for the empirical study
% 2) Describe the research questions

\noindent
This paper's experiments answer these research questions:

\begin{itemize}[leftmargin=*]

    % RQ1: Manual annotation study
    %
    \item {\it RQ1: How does \texttt{TreeNose} perform in different languages?}
        To answer RQ1, we evaluated \texttt{TreeNose}'s code smell detection
        effectiveness by applying it to 9 open-source projects with at least
        10,000 lines of code written in Java, JavaScript, or Python, as shown in
        the top half of Table~\ref{tab:subject_table}. Adopting a manual
        annotation methodology, we answer RQ1 by calculating precision, recall,
        and the F1 score for both \texttt{TreeNose} and three popular
        language-specific tools.

    % RQ2: Code smell percentages
    %
    \item {\it RQ2: How do code smells distribute in various languages?} To
        answer RQ2, we augmented the 9 projects used to answer RQ1 with 7 more,
        including 4 projects that are multi-language in nature, as shown in
        Table~\ref{tab:subject_table}. Using \texttt{TreeNose}'s output, we
        calculated the percentage of the reported code smells, across the chosen
        projects for each supported language and for multi-language projects.

    % RQ3: Code smell prevalence
    %
    \item {\it RQ3: How often do code smells occur in various languages?}
        Again using the combined subject set in Table~\ref{tab:subject_table},
        we answer RQ3 by calculating a discrepancy score that contextualizes the
        prevalence of a code smell by programming language.

\end{itemize}

% Introduce the subjects (note, now using the term "projects" instead of "subjects")

{\bf Projects}. To empirically answer these three RQs, we applied
\texttt{TreeNose} to real-world open-source projects that met the following
criteria: (i) have commits on main branch in the last year; (ii) have at least
10 thousand lines of code; (iii) have more than 1,000 stars on GitHub; and (iv)
have an implementation entirely in one of Java, JavaScript, or Python or,
alternatively, a multi-language implementation using one or more of the three
aforementioned languages.
%
To answer RQ1, we picked the 9 projects in the top half of
Table~\ref{tab:subject_table}.
%
It is worth noting that, even though \texttt{TreeNose} supports both Python
versions 2 and 3, PySmell only supports Python version 2, thereby necessitating
that all of the Python-based projects must be implemented in version 2 of
Python.
%
In support of the answer to RQ2 and RQ3, the bottom half of
Table~\ref{tab:subject_table} includes one more project in each of the three
chosen languages and four additional multi-language projects.
%
For each project, we performed smell detection on the source code of both the
program and its test suite because the test code is important for a developer
to comprehend and maintain~\cite{ML}.

\begin{table}[h]
    % Figures should be centered in the page/column
    \centering
    %
    \caption{
        % The label should appear _inside_ the caption to ensure
        % Latex numbers it correctly. This is a common gotcha!
        %
        % All table labels should start with "tab:"
        % So that the figure file can be found easily, the rest of the
        % table's label should be the same as the filename, as it is
        % in this example:
        %
        \label{tab:subject_table}
        %
        CHARACTERISTICS OF SUBJECT SYSTEMS
    }
    %
    % Depending on the template, some breathing space might need to
    % be added (use \smallskip \medskip etc.) here
    %
    % Or, to save space, you might want to remove space
    % (use a negative \vspace, e.g. \vspace{-1em})
    %
    %
    % Table content goes here. Use this file to specify the
    % table's column headings. The data should be automatically
    % output from a program processing the raw experimental data
    % and should be inputted from another file. This enables
    % the data to change, if for example, the experiment data
    % needs to be updated.
    %
    % Do not use vertical rules. Ensure you use \toprule, \midrule
    % and \bottomrule from the "booktabs" package effectively.
    %
    % Numbers should be right justified (use "r"),
    % text left justified (use "l").
    %
    % For example:
    %
    \renewcommand{\arraystretch}{1.2}
    % \rowcolors{2}{gray!25}{white}
    \begin{tabular}{@{}lllrrrrr@{}}
        \toprule
            % use a new line for each column if needed
            {\bf Project}
            &
            {\bf Organization}
            &
            {\bf Language}
            &
            {\bf Version} 
            &
            {\bf Stars} 
            &
            {\bf KLOC}
            \\



        % Now input the data file Note the filename should not have curly
        % brackets, otherwise latex will generate a warning (see
        % https://tex.stackexchange.com/questions/567985/problems-with-inputtable-tex-hline-after-2020-fall-latex-release
        % as to why)
        
        \bottomrule
        \input table-data/subject_data

    \end{tabular}

    %
    % To save space, you might want to remove space here (use a negative
    % \vspace, e.g. \vspace{-1em})
    \vspace{-1em}
\end{table}


{\bf Methodology}. To answer RQ1, we compared \texttt{TreeNose} with 3
language-specific code smell detection tools, namely PySmell~\cite{Pysmell} for
Python, JScent~\cite{Jscent} for JavaScript, and
DesigniteJava~\cite{DesigniteJava} for Java. All 3 code smell detectors are
language-specific, AST-based code smell detection tools that are open-source;
PySmell was shown to achieve 97.7\% precision in a prior study.
%
In this paper's experiment, we compared TreeNose to the combination of the
other tools, aiming to mimic the scenario in which developers use multiple code
smell detectors to identify language-specific code smells across projects that
use different languages.
%
Since the aforementioned detectors support most of the code smells detected by
\texttt{TreeNose}, we applied them to each project in the top half of
Table~\ref{tab:subject_table}. To quantify the performance of \texttt{TreeNose}
and the other tools, we adopted precision and recall, which are defined as
follows:

\begin{equation}
    \textit{Precision} = \frac{\textit{True Positive}}{\textit{True Positive} + \textit{False Positive}}
\end{equation}

\begin{equation}
    \textit{Recall} = \frac{\textit{True Positive}}{\textit{True Positive} + \textit{False Negative}}
\end{equation}

Combining precision and recall, we also calculated the F1 score, the harmonic
mean of precision and recall, defined as:

\begin{equation}
    \textnormal{F1} = 2 * \frac{\textit{Precision} * \textit{Recall}}{\textit{Precision} + \textit{Recall}}
\end{equation}


\begin{table*}[t]
    % Figures should be centered in the page/column
    \centering
    %
    \caption{
        % The label should appear _inside_ the caption to ensure
        % Latex numbers it correctly. This is a common gotcha!
        %
        % All table labels should start with "tab:"
        % So that the figure file can be found easily, the rest of the
        % table's label should be the same as the filename, as it is
        % in this example:
        %
        \label{tab:annotation_table}
        %
        Precision, Recall, and F1 scores for the TreeNose (TN) and the language-specific (LS) detectors.
    }
    %
    % Depending on the template, some breathing space might need to
    % be added (use \smallskip \medskip etc.) here
    %
    % Or, to save space, you might want to remove space
    % (use a negative \vspace, e.g. \vspace{-1em})
    %
    %
    % Table content goes here. Use this file to specify the
    % table's column headings. The data should be automatically
    % output from a program processing the raw experimental data
    % and should be inputted from another file. This enables
    % the data to change, if for example, the experiment data
    % needs to be updated.
    %
    % Do not use vertical rules. Ensure you use \toprule, \midrule
    % and \bottomrule from the "booktabs" package effectively.
    %
    % Numbers should be right justified (use "r"),
    % text left justified (use "l").
    %
    % For example:
    %
    \renewcommand{\arraystretch}{1.2}
    % \rowcolors{2}{gray!25}{white}
    \begin{tabular}{@{}lrrrrrr@{}}
        \toprule
            % use a new line for each column if needed
            {\bf }
            &
            {\bf TN Precision}
            &
            {\bf LS Precision}
            &
            {\bf TN Recall}
            &
            {\bf LS Recall}
            &
            {\bf TN F1}
            &
            {\bf LS F1}
            \\
            %
        \bottomrule
        \input table-data/annotation_data

    \end{tabular}
    %
    %
    % To save space, you might want to remove space here (use a negative
    % \vspace, e.g. \vspace{-1em})
    \vspace{-1em}
\end{table*}


Since, to our best knowledge, there is no annotated code smell dataset for our
unique combination of projects implemented both in the three languages and in a
multi-language fashion, we designed a manual annotation study inspired by the
design of Palomba~\etal's experiment~\cite{Palomba2018}.
%
To answer RQ1, we first applied both \texttt{TreeNose} and the other tools with
their default thresholds to analyze the projects in the top half of
Table~\ref{tab:subject_table}.
%
Facilitating a comparison between the presented tool and the language-specific
smell detectors, we first set \texttt{TreeNose} as the ``baseline'' detector and
randomly sampled 5 code segments for each smell.
%
Next, the authors of this paper independently inspected these flagged code
smells and determined whether, from a developer's perspective, they were valid
not.
%
The authors then discussed their independent classifications until a consensus
emerged as to whether the detected code smells were valid or
not~\cite{Cruzes2011}.
%
We designated a \textit{True Positive} for \texttt{TreeNose} when the authors
and \texttt{TreeNose} classified the randomly selected code as a smell and a
\textit{False Positive} for \texttt{TreeNose} when we did not classify it as a
code smell and yet the presented tool did.
%
When \texttt{TreeNose} was the baseline technique, we designated a \textit{False
Negative} for the language-specific detector when it did not classify the source
code segment as a smell but we did.
%
At this step in the manual annotation study, we calculated the precision for
both \texttt{TreeNose} and the language-specific detector and both recall and
the F1 score for the language-specific detector.

The next step of the manual annotation study set the language-specific tool as
the baseline technique and randomly sampled 5 source code segments that it
claimed were a code smell.
%
Analogous to the prior setup, this configuration enabled us to calculate
precision for both the language-specific smell detector and \texttt{TreeNose}
and to additionally calculate recall and the F1 score for \texttt{TreeNose}.
%
In the absence of an existing data set with confirmed code smell annotations,
this design for the manual annotation study facilitated the indirect comparison
between \texttt{TreeNose} and the other language-specific detectors.
%
Finally, it is important to note that not all code smells detected by
\texttt{TreeNose} are detected by the other tools. For those code smells, we
skipped them in the calculation of the F1 score for \texttt{TreeNose} because
there is no detection results from the other tools, and labeled them as not
detected in the calculation of the F1 score of the other tools. We made this
decision because the other tools ``silently failed'' to detect these code
smells.

% Cut this content:

% With those metrics, we calculated the precision, recall, and
% F1 score of \texttt{TreeNose}. Then, we conducted the same study with the
% samples from \texttt{TreeNose} to calculate the F1 score of the other tools.
% Finally, we utilized the F1 scores of \texttt{TreeNose} and the other tools to
% compare their performances.

% We, furthermore, conducted such a study with 2 internal developers to manually
% verify code smells. Specifically, we built two spreadsheets for the study. One
% for the code smell reports generated by \texttt{TreeNose}, and another for the
% code smell reports generated by the other tools.

% Within the code smell reports
% generated by the other tools, we randomly selected 5 samples out of each code
% smell in each programming language system, and then put those samples in the
% spreadsheet with their related source code and a column for the detection
% decision of \texttt{TreeNose}.

% By this approach, we can compare the decisions between the \texttt{TreeNose}
% and us, where both can potentially agree or disagree with the other detectors.

% Change 20 to 16 to match up with the Table II Remove "up to" 16 to make it
% precise
%
To answer both RQ2 and RQ3, we applied \texttt{TreeNose} to all of the 16
open-source projects in Table~\ref{tab:subject_table}.
%
% The phrase "the combination of the selected languages" is redundant.
%
% combination of them.
%
% Redundant sentence We implemented \texttt{TreeNose} in the projects and
% analyzed the occurrence of each code smell in the projects.
%
Analyzing the code smell reports of \texttt{TreeNose} helped to characterize the
distribution and prevalence of the selected code smells in different programming
languages, and the target languages' tendencies to contain specific code smells.
%
After this step, we calculated the evaluation metrics reported in
Tables~\ref{tab:percentage_table} and~\ref{tab:occurrence_table}.

% RQ2, we calculate the number of each code smell in individual projects
% ($s_{k}$) and divide it by the total number of code smells in that project
% ({$S$}) to calculate the project-level percentage of the code smell. Then we
% the sum of the project-level percentages in the projects in the language, and
% divide it by the number of the projects in the language ($N_r$) to calculate
% the percentage of the code smell in the language ($P_{k}$).

% In the experiment of the RQ2, we calculate the percentage of each code smell
% out of all the selected code smells as follows:

% Specify the purpose of Equation 4 at beginning to reinforce the relationship
% between the equation and the RQ2

We calculated the distribution of each code smell, denoted $P_{s}$, with the
percentage formula given by Equation~\ref{eq:percentage}. To minimize the bias
introduced by the different project sizes (i.e., KLOC in
Table~\ref{tab:subject_table}), the percentage of the code smell in each project
was calculated and then averaged for the projects implemented with that
programming language.
%
We divided the number of each code smell in a project ($m_{k}$) by the number of
all kinds of code smells in that project ({$M_{k}$}), thus calculating the
project-level percentage. We then took the sum of the project-level percentages
and divided it by the number of the projects in the language ($k$) to calculate
$P_{s}$. The results associated with these percentages are shown in
Table~\ref{tab:percentage_table}.

% \vspace{-0.5em}

\begin{equation}
P_{s} = \left( \sum_{1}^{k}\frac{{m_{k}}}{M_{k}} \right) \div {k} \times 100\%
\label{eq:percentage}
\end{equation}

% As for the RQ3, we would like to quantify the prevalence of the code smells as
% the occurrence of the code smells in each 1000 lines of code on average
% ($C_{s}$). To calculate it, we divide the number of each code smell in
% individual projects ($s_{k}$) by the total number of lines in the project
% ($N_{l}$) to calculate the project-level prevalence of the code smell, and
% then take the average of the occurrences in each project. Finally by taking the
% average of the occurrences in the projects in the language, we calculate the
% prevalence of the code smell in the language as follows:

To answer RQ3, we quantified the prevalence of a code smell as the average
occurrence of the code smell in each 1000 lines of code, denoted as $C_{s}$ and
shown in Equation~\ref{eq:prevalence}. To compute this evaluation metric, we
divided the count of a code smell in one project ($m_{k}$) by the total number
of lines in the project ($N_{k}$), thereby giving the project-level occurrence
of the code smell, and then calculate the language-level occurrence score
($C_{s}$) by averaging the project-level occurrences in the projects in the
language and multiplying the result by 1000.

% \vspace{-0.5em}

\begin{equation}
C_{s} = \left( \sum_{1}^{k}{\frac{m_{k}}{N_{k}}} \right) \div k \times 1000
\label{eq:prevalence}
\end{equation}

% With the prevalence score $C_{s}$, we want to quantify how much one language
% outperforms the others. Therefore, we utilized the discrepancy formula to
% calculate the occurrence discrepancy score. We compared the occurrence of one
% language ($C_{s}$) against the same code smell in the combination of the other
% programming languages ($C_{O}$). We calculate the occurrence discrepancy score
% (D) of each code smell in the projects as follows:

Leveraging the code smell occurrence score $C_{s}$ from
Equation~\ref{eq:prevalence}, we calculated the discrepancy formula shown in the
Equation~\ref{eq:discrepancy} to quantify the classification differences between
the selected programming languages and thereby answer RQ3.
%
We calculated the difference with the occurrence of a code smell in one language
($C_{s}$) minus the same code smell in the combination of the other languages
($C_{O}$), and calculated the occurrence discrepancy score (D) by dividing
$C_{O}$. The results associated with these discrepancy scores are shown in
Table~\ref{tab:occurrence_table}.

% \vspace{-1.0em}

\begin{equation}
    D = (C_{O} - C_{s}) \div C_{O}
    \label{eq:discrepancy}
\end{equation}

In the Equation~\ref{eq:discrepancy}, a positive score indicates that the code
smell occurs less frequently in the language than the other languages, while a
negative score indicates that the code smell occurs more frequently in the
language than the other languages.
%
To represent this clearly in Table~\ref{tab:occurrence_table}, we use $\uparrow$
to designate a positive value for this metric and $\downarrow$ for a negative
one.
%
With that said, the discrepancy score has a range of negative infinity to 1,
where 1 means the code smell doesn't occur in the language at all and 0 means
the code smell occurs in the language as frequently as the other languages.

{\bf Threats to Validity}.
%
Although it is investigated by numerous prior empirical studies, the concept of
a code smell is inherently subjective~\cite{Fowler_Beck}.
%
For instance, we noted that various code smell detection tools use different
metrics and thresholds to detect the same code smell. Thus, the metrics and
thresholds for \texttt{TreeNose}, given in
Table~\ref{tab:metrics-and-thresholds}, may not align with the other tools and
may not be ideal for detecting code smells in the chosen projects.
%
% Similarly, the selection of metrics and thresholds in \texttt{TreeNose} may
% not be ideal for detecting selected code smells.
%
To mitigate these threats, this paper's authors performed a manual annotation
study to evaluate \texttt{TreeNose}'s effectiveness at automatically detecting
code smells.

% Remove the sentence cuz it's redundant with the end of last paragraph
% """Another threat is related to the annotation study. To evaluate the
% performance of \texttt{TreeNose}, we conducted an annotation study with 2
% developers. """
%
Another threat relates to the fact that the authors conducted the annotation
study and their consensus may not align with the views of all developers.
%
Moreover, to ensure feasibility, the methodology for the manual annotation study
only considered a limited number of samples from 9 of the 17 projects.
%
This means that, even though these significant open-source projects are large,
actively used, and well-maintained, the results may still not be generalizable
to all software projects in the target language. To address this threat, we plan
to conduct more studies with both more annotators and projects in the future.

% Note: in my view, the third threat is not a threat to validity but a
% limitation, so we may delete it?

% The third threat is related to the limitation of the selected code smell
% detectors. They didn't detect all the selected code smells, and the results may
% be biased by the missing code smells. To mitigate the threats, we plan to
% implement more code smell detectors in the future to compare the performance of
% \texttt{TreeNose} with more detectors.


\begin{table}[h]
    % Figures should be centered in the page/column
    \centering
    %
    \caption{
        % The label should appear _inside_ the caption to ensure
        % Latex numbers it correctly. This is a common gotcha!
        %
        % All table labels should start with "tab:"
        % So that the figure file can be found easily, the rest of the
        % table's label should be the same as the filename, as it is
        % in this example:
        %
        \label{tab:percentage_table}
        %
        PERCENTAGE OF CODE SMELLS IN PROGRAMMING LANGUAGE SYSTEMS
    }
    %
    % Depending on the template, some breathing space might need to
    % be added (use \smallskip \medskip etc.) here
    %
    % Or, to save space, you might want to remove space
    % (use a negative \vspace, e.g. \vspace{-1em})
    %
    %
    % Table content goes here. Use this file to specify the
    % table's column headings. The data should be automatically
    % output from a program processing the raw experimental data
    % and should be inputted from another file. This enables
    % the data to change, if for example, the experiment data
    % needs to be updated.
    %
    % Do not use vertical rules. Ensure you use \toprule, \midrule
    % and \bottomrule from the "booktabs" package effectively.
    %
    % Numbers should be right justified (use "r"),
    % text left justified (use "l").
    %
    % For example:
    %
    \renewcommand{\arraystretch}{1.2}
    % \rowcolors{2}{gray!25}{white}
    \begin{tabular}{@{}lrrrrrr@{}}
        \toprule
            % use a new line for each column if needed
            {\bf }
            &
            {\bf CC} 
            &
            {\bf LC}  
            &
            {\bf LM}
            &
            {\bf LMC}
            &
            {\bf LPL}
            \\



        % Now input the data file Note the filename should not have curly
        % brackets, otherwise latex will generate a warning (see
        % https://tex.stackexchange.com/questions/567985/problems-with-inputtable-tex-hline-after-2020-fall-latex-release
        % as to why)
        
        \bottomrule
        \input table-data/percentage_data

    \end{tabular}

    %
    % To save space, you might want to remove space here (use a negative
    % \vspace, e.g. \vspace{-1em})
    \vspace{-1em}
\end{table}


% Test code may be different than source code of the program

The last threat is due to the decision to include the test code --- along with
the source code of the program under test --- in the analysis used to answer
each research question. While test code is important for developers to
comprehend and maintain in the software development~\cite{ML}, it may adhere to
different conventions and styles from the program's source
code~\cite{Kapfhammer2004,Kapfhammer2010}. This means that the detection
thresholds of Table~\ref{tab:metrics-and-thresholds}, which are mainly designed
for a program's source code, may not be ideal for the test code. To mitigate
this threat, we plan to conduct future studies that excluding the test code.
