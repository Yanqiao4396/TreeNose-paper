
\input{tables/percentage_table}
\section{Results}
\label{sec:results}

In this section, we present the results of our evaluation.


Table~\ref{tab:annotation_table} shows the confusion matrixes of \texttt{TreeNose} and the other tools in the annotation study. TN represents \texttt{TreeNose}, and OD represents the other tools.
On the average, \texttt{TreeNose} significantly outperformed the other tools in terms of precision, recall, and F1 score.
With the 0.94 F1 score, we conclude that \texttt{TreeNose} achieved a high level of accuracy in detecting the code smells.
When there is no value in the table, it means it's not possible to calculate the recall. For example, 
\texttt{TreeNose} flagged all the positive when we selected samples from the \texttt{TreeNose} detection. Therefore, it's not
possible to calculate the recall or the F1 score of \texttt{TreeNose} in this environment. The same applies to the OD-selected samples.
Another interesting observation is that the OD group achieves 1.0 precision in the \texttt{TreeNose}-selected samples, but the recall is down to 0.32,
which results in a low F1 score of 0.48. This indicates that the OD group can hardly detect all the code smells in the \texttt{TreeNose}-selected samples.

Table~\ref{tab:percentage_table} discloses the percentage of the code smells in each programming language system. Looking at the table, we can see Complex Conditional (CC)
counts for 42\% of code smells on average, indicating that it is the most common code smell across programming languages. 
the percentage of Long Class (LC) and Long Method (LM) vary significantly in the Java and JavaScript systems. In the Java system,
LC occurs 10 times more than LM, while in the JavaScript system it's the opposite, LM occurs 30 times more than LC. 
This indicates that the programming languages have huge different tendencies in terms of code smells.

\input{tables/occurrence_table}

Table~\ref{tab:occurrence_table} shows the occurrence discrepancy scores of the code smells in the systems. Since no programming language contains positive values in every code smell, 
we conclude no programming language outperforms the others in every code smell.
By looking at each code smell, we also see programming languages have strong tendencies in some code smells with at least 1 time worse performance than others. 
For example, the Java system has a strong tendency in the Long Class (LC) with 3 times worse performance than others, 
while the JavaScript system has a strong tendency in the Long Method (LM) code smell. Python has a strong tendency in the Long Parameter List (LPL).
Multi-Language system, on the other hand, has weaker tendencies in code smells compared to the other systems with zero negative values less than - 1.0.

