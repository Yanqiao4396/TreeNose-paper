% Present the answer to RQ1


\begin{table*}[t]
    % Figures should be centered in the page/column
    \centering
    %
    \caption{
        % The label should appear _inside_ the caption to ensure
        % Latex numbers it correctly. This is a common gotcha!
        %
        % All table labels should start with "tab:"
        % So that the figure file can be found easily, the rest of the
        % table's label should be the same as the filename, as it is
        % in this example:
        %
        \label{tab:occurrence_table}
        %
        DISCREPANCY SCORES OF CODE SMELLS IN PROGRAMMING LANGUAGE SYSTEMS
    }
    %
    % Depending on the template, some breathing space might need to
    % be added (use \smallskip \medskip etc.) here
    %
    % Or, to save space, you might want to remove space
    % (use a negative \vspace, e.g. \vspace{-1em})
    %
    %
    % Table content goes here. Use this file to specify the
    % table's column headings. The data should be automatically
    % output from a program processing the raw experimental data
    % and should be inputted from another file. This enables
    % the data to change, if for example, the experiment data
    % needs to be updated.
    %
    % Do not use vertical rules. Ensure you use \toprule, \midrule
    % and \bottomrule from the "booktabs" package effectively.
    %
    % Numbers should be right justified (use "r"),
    % text left justified (use "l").
    %
    % For example:
    %
    \renewcommand{\arraystretch}{1.2}
    % \rowcolors{2}{gray!25}{white}
    \begin{tabular}{@{}lrrrrrrr@{}}
        \toprule
            % use a new line for each column if needed
            {\bf }
            &
            {\bf CC} 
            &
            {\bf LC}  
            &
            {\bf LM}
            &
            {\bf LMC}
            &
            {\bf LPL}
            \\



        % Now input the data file Note the filename should not have curly
        % brackets, otherwise latex will generate a warning (see
        % https://tex.stackexchange.com/questions/567985/problems-with-inputtable-tex-hline-after-2020-fall-latex-release
        % as to why)
        
        \bottomrule
        \input table-data/occurrence_data

    \end{tabular}

    %
    % To save space, you might want to remove space here (use a negative
    % \vspace, e.g. \vspace{-1em})
    \vspace{-1em}
\end{table*}


{\bf Answering RQ1}. Table~\ref{tab:annotation_table} furnishes the evaluation
metrics arising from the manual annotation study of \texttt{TreeNose} and the
other language-specific tools.
%
In this table, TN represents \texttt{TreeNose}, and LS represents a
language-specific tool.
%
When there is no value in this table, it means that it is not possible to
calculate the recall, per the methodology described in
Section~\ref{sec:evaluation}.
%
For example, when \texttt{TreeNose} was the baseline technique, it suggested
all of the code smells from which we randomly sampled and thus it is not
possible to calculate the recall or the F1 score of \texttt{TreeNose} in this
environment, with the same applying to the LS-selected samples.
%
Overall, Table~\ref{tab:annotation_table} shows that \texttt{TreeNose}
significantly outperformed the other tools in terms of precision, recall, and
F1 score.
%
For instance, it exhibits an F1 score of 0.94 compared to one of 0.48 for the
language-specific detectors.
%
Interestingly, the language-specific detectors achieve a precision of 1.0 for
the \texttt{TreeNose}-selected samples, but their recall is down to 0.32,
resulting in the aforementioned low F1 score of 0.48.

% This indicates that the LS group can hardly detect all the code smells in the
% \texttt{TreeNose}--selected samples.

{\bf Answering RQ2}. Table~\ref{tab:percentage_table} furnishes the percentage
of the code smells in each programming language. This table shows that the
Complex Conditional (CC) code smell accounts for 46\% of code smells on
average, indicating that it is the most common code smell across programming
languages.
%
The table also reveals that the percentage of Long Class (LC) and Long Method
(LM) vary significantly in the Java and JavaScript projects. In the Java
projects, LC occurs 7 times more than LM, while in the JavaScript system it is
the opposite as LM occurs 30 times more than LC. This suggests that programming
languages have different code smell tendencies.

{\bf Answering RQ3}. Table~\ref{tab:occurrence_table} shows the occurrence
discrepancy scores of the code smells across the chosen languages. Since no
programming language contains positive values in every code smell, we conclude
that no programming language ``outperforms'' the others for every code smell.
By looking at each code smell, this table shows that programming languages have
strong tendencies for certain code smells.
%
For example, the Java projects have a strong tendency to exhibit Long Class
(LC), with 1.24 times worse performance and Long Message Chain (LMC) with 3.56
worse performance than others, while the JavaScript system has a strong
tendency in the Long Method (LM) code smell. Python has a strong tendency to
have Long Parameter List (LPL) smells. However, multi-Language projects have
less pronounced code smell tendencies compared to the single-language projects,
with values for the discrepancy metric $D$ ranging between -0.23 and 0.70.
