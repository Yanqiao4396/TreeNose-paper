% Present the answer to RQ1


\begin{table*}[t]
    % Figures should be centered in the page/column
    \centering
    %
    \caption{
        % The label should appear _inside_ the caption to ensure
        % Latex numbers it correctly. This is a common gotcha!
        %
        % All table labels should start with "tab:"
        % So that the figure file can be found easily, the rest of the
        % table's label should be the same as the filename, as it is
        % in this example:
        %
        \label{tab:occurrence_table}
        %
        DISCREPANCY SCORES OF CODE SMELLS IN PROGRAMMING LANGUAGE SYSTEMS
    }
    %
    % Depending on the template, some breathing space might need to
    % be added (use \smallskip \medskip etc.) here
    %
    % Or, to save space, you might want to remove space
    % (use a negative \vspace, e.g. \vspace{-1em})
    %
    %
    % Table content goes here. Use this file to specify the
    % table's column headings. The data should be automatically
    % output from a program processing the raw experimental data
    % and should be inputted from another file. This enables
    % the data to change, if for example, the experiment data
    % needs to be updated.
    %
    % Do not use vertical rules. Ensure you use \toprule, \midrule
    % and \bottomrule from the "booktabs" package effectively.
    %
    % Numbers should be right justified (use "r"),
    % text left justified (use "l").
    %
    % For example:
    %
    \renewcommand{\arraystretch}{1.2}
    % \rowcolors{2}{gray!25}{white}
    \begin{tabular}{@{}lrrrrrrr@{}}
        \toprule
            % use a new line for each column if needed
            {\bf }
            &
            {\bf CC} 
            &
            {\bf LC}  
            &
            {\bf LM}
            &
            {\bf LMC}
            &
            {\bf LPL}
            \\



        % Now input the data file Note the filename should not have curly
        % brackets, otherwise latex will generate a warning (see
        % https://tex.stackexchange.com/questions/567985/problems-with-inputtable-tex-hline-after-2020-fall-latex-release
        % as to why)
        
        \bottomrule
        \input table-data/occurrence_data

    \end{tabular}

    %
    % To save space, you might want to remove space here (use a negative
    % \vspace, e.g. \vspace{-1em})
    \vspace{-1em}
\end{table*}


{\bf Answering RQ1}. Table~\ref{tab:annotation_table} shows the confusion
matrixes of \texttt{TreeNose} and the other tools in the annotation study. TN
represents \texttt{TreeNose}, and LS represents the other tools. On the
average, \texttt{TreeNose} significantly outperformed the other tools in terms
of precision, recall, and F1 score. With the 0.94 F1 score, we conclude that
\texttt{TreeNose} achieved a high level of accuracy in detecting the code
smells. When there is no value in the table, it means it's not possible to
calculate the recall. For example, \texttt{TreeNose} flagged all the positive
when we selected samples from the \texttt{TreeNose} detection. Therefore, it's
not possible to calculate the recall or the F1 score of \texttt{TreeNose} in
this environment. The same applies to the LS-selected samples. Another
interesting observation is that the LS group achieves 1.0 precision in the
\texttt{TreeNose}-selected samples, but the recall is down to 0.32, which
results in a low F1 score of 0.48. This indicates that the LS group can hardly
detect all the code smells in the \texttt{TreeNose}--selected samples.

{\bf Answering RQ2}. Table~\ref{tab:percentage_table} discloses the percentage
of the code smells in each programming language system. Looking at the table,
we can see Complex Conditional (CC) counts for 46\% of code smells on average,
indicating that it is the most common code smell across programming languages.
the percentage of Long Class (LC) and Long Method (LM) vary significantly in
the Java and JavaScript systems. In the Java system, LC occurs 7 times more
than LM, while in the JavaScript system it's the opposite, LM occurs 30 times
more than LC. This indicates that the programming languages have huge different
tendencies in terms of code smells.

{\bf Answering RQ3}. Table~\ref{tab:occurrence_table} shows the occurrence
discrepancy scores of the code smells in the systems. Since no programming
language contains positive values in every code smell, we conclude no
programming language outperforms the others in every code smell. By looking at
each code smell, we also see programming languages have strong tendencies in
some code smells with at least 1 time worse performance than others. For
example, the Java system has a strong tendency in the Long Class (LC) with 1.24
times worse performance and Long Message Chain (LMC) with 3.56 worse performance than others, 
while the JavaScript system has a strong
tendency in the Long Method (LM) code smell. Python has a strong tendency in
the Long Parameter List (LPL). Multi-Language system, on the other hand, has
weaker tendencies in code smells compared to the other systems with zero
negative values less than - 1.0.
