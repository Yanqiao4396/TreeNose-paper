\section{Results}
\label{sec:results}

In this section, we present the results of our evaluation. To avoid redundancy, we combined the evaluation data of the RQ2 and the RQ3 into a single table.


Table~\ref{tab:annotation_table} shows the confusion matrixes of \texttt{TreeNose} and the other tools in the annotation study. TN represents \texttt{TreeNose}, and OD represents the other tools.
On the average, \texttt{TreeNose} significantly outperformed the other tools in terms of precision, recall, and F1-score.
With the 0.94 F1 score, we conclude that \texttt{TreeNose} achieved a high level of accuracy in detecting the code smells.
When there is no value in the table, it means it's not possible to calculate the recall. For example, 
\texttt{TreeNose} flagged all the positive when we selected samples from the \texttt{TreeNose} detection. Therefore, it's not
possible to calculate the recall or the F1 score of \texttt{TreeNose} in this environment. The same applies to the OD-selected samples.
Another interesting observation is that the OD group achieves 1.0 precision in the \texttt{TreeNose}-selected samples, but the recall is down to 0.32,
which results in a low F1 score of 0.48. This indicates that the OD group is not able to detect all the code smells in the \texttt{TreeNose}-selected samples.

Table~\ref{tab:percentage_table} discloses the percentage of the code smells in each programming language system. Looking at the table, we can see Complex Conditional (CC)
counts for almost 1/3 of the code smells in each system, indicating that it is the most common code smell across programming languages. 
Long Class (LC) and Long Method (LM) behave differently in the Java and JavaScript systems. In the Java system,
LC occurs 30 times more than LM, while in the JavaScript system it's the opposite, LM occurs 10 times more than LC. This indicates that the programming languages
have different tendencies in terms of code smells.

Table~\ref{tab:occurrence_table} shows the occurrence discrepancy scores of the code smells in the systems. It indicates that no programming language outperforms the others in every code smell,
meaning that each programming language has its own strengths and weaknesses in terms of code smells. For example, Java has the highest score in the LM (Long Method) code smell,
 while Python has the highest score in the LMC (Long Message Chain) code smell. We also learned that certain code smells significantly vary in occurrence across programming languages.
 Especially, Java performs 3 times worse than the other programming languages in the LC (Long Class) code smell. 
 This is aligned with the result from Table~\ref{tab:percentage_table}., where we observed that the LC code smell is the most common code smell in the Java system.



\begin{table*}[t]
    % Figures should be centered in the page/column
    \centering
    %
    \caption{
        % The label should appear _inside_ the caption to ensure
        % Latex numbers it correctly. This is a common gotcha!
        %
        % All table labels should start with "tab:"
        % So that the figure file can be found easily, the rest of the
        % table's label should be the same as the filename, as it is
        % in this example:
        %
        \label{tab:annotation_table}
        %
        Precision, Recall, and F1 scores for the TreeNose (TN) and the language-specific (LS) detectors.
    }
    %
    % Depending on the template, some breathing space might need to
    % be added (use \smallskip \medskip etc.) here
    %
    % Or, to save space, you might want to remove space
    % (use a negative \vspace, e.g. \vspace{-1em})
    %
    %
    % Table content goes here. Use this file to specify the
    % table's column headings. The data should be automatically
    % output from a program processing the raw experimental data
    % and should be inputted from another file. This enables
    % the data to change, if for example, the experiment data
    % needs to be updated.
    %
    % Do not use vertical rules. Ensure you use \toprule, \midrule
    % and \bottomrule from the "booktabs" package effectively.
    %
    % Numbers should be right justified (use "r"),
    % text left justified (use "l").
    %
    % For example:
    %
    \renewcommand{\arraystretch}{1.2}
    % \rowcolors{2}{gray!25}{white}
    \begin{tabular}{@{}lrrrrrr@{}}
        \toprule
            % use a new line for each column if needed
            {\bf }
            &
            {\bf TN Precision}
            &
            {\bf LS Precision}
            &
            {\bf TN Recall}
            &
            {\bf LS Recall}
            &
            {\bf TN F1}
            &
            {\bf LS F1}
            \\
            %
        \bottomrule
        \input table-data/annotation_data

    \end{tabular}
    %
    %
    % To save space, you might want to remove space here (use a negative
    % \vspace, e.g. \vspace{-1em})
    \vspace{-1em}
\end{table*}



\begin{table}[h]
    % Figures should be centered in the page/column
    \centering
    %
    \caption{
        % The label should appear _inside_ the caption to ensure
        % Latex numbers it correctly. This is a common gotcha!
        %
        % All table labels should start with "tab:"
        % So that the figure file can be found easily, the rest of the
        % table's label should be the same as the filename, as it is
        % in this example:
        %
        \label{tab:percentage_table}
        %
        PERCENTAGE OF CODE SMELLS IN PROGRAMMING LANGUAGE SYSTEMS
    }
    %
    % Depending on the template, some breathing space might need to
    % be added (use \smallskip \medskip etc.) here
    %
    % Or, to save space, you might want to remove space
    % (use a negative \vspace, e.g. \vspace{-1em})
    %
    %
    % Table content goes here. Use this file to specify the
    % table's column headings. The data should be automatically
    % output from a program processing the raw experimental data
    % and should be inputted from another file. This enables
    % the data to change, if for example, the experiment data
    % needs to be updated.
    %
    % Do not use vertical rules. Ensure you use \toprule, \midrule
    % and \bottomrule from the "booktabs" package effectively.
    %
    % Numbers should be right justified (use "r"),
    % text left justified (use "l").
    %
    % For example:
    %
    \renewcommand{\arraystretch}{1.2}
    % \rowcolors{2}{gray!25}{white}
    \begin{tabular}{@{}lrrrrrr@{}}
        \toprule
            % use a new line for each column if needed
            {\bf }
            &
            {\bf CC} 
            &
            {\bf LC}  
            &
            {\bf LM}
            &
            {\bf LMC}
            &
            {\bf LPL}
            \\



        % Now input the data file Note the filename should not have curly
        % brackets, otherwise latex will generate a warning (see
        % https://tex.stackexchange.com/questions/567985/problems-with-inputtable-tex-hline-after-2020-fall-latex-release
        % as to why)
        
        \bottomrule
        \input table-data/percentage_data

    \end{tabular}

    %
    % To save space, you might want to remove space here (use a negative
    % \vspace, e.g. \vspace{-1em})
    \vspace{-1em}
\end{table}



\begin{table*}[t]
    % Figures should be centered in the page/column
    \centering
    %
    \caption{
        % The label should appear _inside_ the caption to ensure
        % Latex numbers it correctly. This is a common gotcha!
        %
        % All table labels should start with "tab:"
        % So that the figure file can be found easily, the rest of the
        % table's label should be the same as the filename, as it is
        % in this example:
        %
        \label{tab:occurrence_table}
        %
        DISCREPANCY SCORES OF CODE SMELLS IN PROGRAMMING LANGUAGE SYSTEMS
    }
    %
    % Depending on the template, some breathing space might need to
    % be added (use \smallskip \medskip etc.) here
    %
    % Or, to save space, you might want to remove space
    % (use a negative \vspace, e.g. \vspace{-1em})
    %
    %
    % Table content goes here. Use this file to specify the
    % table's column headings. The data should be automatically
    % output from a program processing the raw experimental data
    % and should be inputted from another file. This enables
    % the data to change, if for example, the experiment data
    % needs to be updated.
    %
    % Do not use vertical rules. Ensure you use \toprule, \midrule
    % and \bottomrule from the "booktabs" package effectively.
    %
    % Numbers should be right justified (use "r"),
    % text left justified (use "l").
    %
    % For example:
    %
    \renewcommand{\arraystretch}{1.2}
    % \rowcolors{2}{gray!25}{white}
    \begin{tabular}{@{}lrrrrrrr@{}}
        \toprule
            % use a new line for each column if needed
            {\bf }
            &
            {\bf CC} 
            &
            {\bf LC}  
            &
            {\bf LM}
            &
            {\bf LMC}
            &
            {\bf LPL}
            \\



        % Now input the data file Note the filename should not have curly
        % brackets, otherwise latex will generate a warning (see
        % https://tex.stackexchange.com/questions/567985/problems-with-inputtable-tex-hline-after-2020-fall-latex-release
        % as to why)
        
        \bottomrule
        \input table-data/occurrence_data

    \end{tabular}

    %
    % To save space, you might want to remove space here (use a negative
    % \vspace, e.g. \vspace{-1em})
    \vspace{-1em}
\end{table*}

