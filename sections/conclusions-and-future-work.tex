\section{Conclusions and Future Work}

In this paper, we presented \texttt{TreeNose}, a language-independent code
smell detection tool. \texttt{TreeNose} uses TreeSitter AST to detect code
smells without the need for language-specific rules. \texttt{TreeNose} can
detect 5 code smells: Complex Conditional (CC), Long Class (LC), Long Method
(LM), Long Method Chain (LMC), and Long Parameter List (LPL). We evaluated
\texttt{TreeNose} with an annotation study in three programming language
system, i.e., Java, JavaScript, and Python, and compared it with other
language-specific tools in a set of high-quality open-source projects in target
programming languages. As a result, \texttt{TreeNose} achieved a high level of
accuracy with 0.94 F1 score in detecting selected code smells. As a comparison,
the combination of other tools achieved 0.48 F1 score. We concluded
\texttt{TreeNose} is an effective tool for detecting code smells across
programming language systems. We also analyzed the distribution and prevalence
of code smells in the programming language systems with \texttt{TreeNose}. As
results of the analysis, we found that 1. Complex Conditional (CC) is the most
common code smell across programming languages with 42\% average proportion. 2.
Programming languages have strong tendencies in some code smells with at least
1 time worse performance than others. 3. systems written in multi-language have
weaker tendencies in code smells compared to the single language systems with
zero negative values less than - 1.0.

In the future, we plan to extend \texttt{TreeNose} to detect more code smells
and evaluate it with more programming language systems. As for annotation, we
plan to conduct studies with more participants and subjects to increase the
reliability of the results. We also wish to implement \texttt{TreeNose} in a
real-world software development environment to see how it can help developers
maintain the quality of their codebase.
