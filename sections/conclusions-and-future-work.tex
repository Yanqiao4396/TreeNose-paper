\section{Conclusions and Future Work}

% Removed this text and replace it with content that is a little shorter:

% with an annotation study in three programming language system, i.e., Java,
% JavaScript, and Python, and compared it with other language-specific tools in
% a set of high-quality open-source projects in target programming languages.

% As a result, \texttt{TreeNose} achieved a high level of
% accuracy with 0.94 F1 score in detecting selected code smells. As a comparison,
% the combination of other tools achieved 0.48 F1 score. We concluded
% \texttt{TreeNose} is an effective tool for detecting code smells across
% programming language systems. We also analyzed the distribution and prevalence
% of code smells in the programming language systems with \texttt{TreeNose}.

% As results of the analysis, we found that 1. Complex Conditional (CC) is the
% most common code smell across programming languages with 42\% average
% proportion. 2. Programming languages have strong tendencies in some code smells
% with at least 1 time worse performance than others. 3. systems written in
% multi-language have weaker tendencies in code smells compared to the single
% language systems with zero negative values less than - 1.0.

% Review the TreeNose tool, experimental setup, and the results

This paper presents \texttt{TreeNose}, a language-independent code smell
detection tool that leverages Treesitter parsers. \texttt{TreeNose} detects 5
code smells that other language-specific tools (e.g., PySmell and JScent)
commonly support: CC, LC, LM, LMC, and LPL, as defined in
Section~\ref{sec:approach}.
%
Using a manual annotation study, we compared the scores arising from
\texttt{TreeNose}'s classifications to those made with the smell detectors for
Java, JavaScript, and Python,
%
ultimately showing that the language-specific tools yielded an F1 score of 0.48
in comparison to a score of 0.94 from \texttt{TreeNose}.

Building on this compelling result and \texttt{TreeNose}'s capability to
uniformly detect code smells both across projects in different languages and in
multi-language projects, we further used the presented tool to study the
prevalence of code smells.
%
This analysis revealed interesting findings, including the fact that, among the
studied code smells, CC is the most common across the studied projects,
comprising 42\% of the detected smells, including 50\% of those in
multi-language projects.

% Define the future work now that we've shown the tool is so great

In the future, we plan to extend \texttt{TreeNose} to detect more code smells
and evaluate it with more programming language systems. As for annotation, we
plan to conduct studies with more participants and subjects to increase the
reliability of the results. We also wish to implement \texttt{TreeNose} in a
real-world software development environment to see how it can help developers
maintain the quality of their codebase.
