\section{Introduction}
\label{sec:introduction}

% Always edit your text without the use of word-wrap, like the paragraphs in
% this section, because:
%
% 1. Hard-breaks make it easier to track changes in version control.
% 2. It is easier to insert comments explaining some aspect of your writing
%    mid-sentence
% 3. You can break up the structure of your sentence across multiple lines so
%    that it is easier to edit later.

% If you are having trouble structuring or phrasing your text, try summarising
% the key point of what you want to get across in a leading comment, like this:

% This section sets the stage for the paper

Code smells are a series of code-design-related defects. They decrease the
readability \cite{5741260} and maintainability \cite{6392174} \cite{6405287} of software projects,
which block the potential for the future maintenance \cite{Fowler_Beck}.

Due to the negative impact of code smells on software projects,
developers acknowledge the importance of detecting them. The manual detection in
software projects is is an error-prone and resource-consuming process \cite{DetectingDefectsInObject}. 
Therefore, code smell detection tools have been developed to automate this process.
Different Code smell detection tools rely on various detection strategies,
such as Abstract Syntax Tree (AST) Analysis, Machine Learning, and Static Analysis.
Some of the popular code smell detection tools like PMD \cite{PMD} and CheckStyle \cite{CheckStyle}, have become one of the essential
tools in the modern software development process.

One of the characteristics of modern software projects is the use of multiple programming languages \cite{723183}. The combinations of
programming languages allow developers to reply on functionalities and libraries that are not available in a single programming language.
However, the complexity of multi-language software projects increases the difficulty of project comprehension and maintenance \cite{10.1109/SCAM.2012.11} \cite{7476675} \cite{7396422}.
Along with the complexity, the multi-language software projects also introduce the challenge of code smell detection. 
The existing code smell detection tools are designed to detect code smells in a single programming language. Therefore, developers have to configure multiple 
code smell detection tools in multi-language software projects.

On the other end of the spectrum, most code smells are language-independent. For example,
the Long Method, one of the most prevalent code smells \cite{developersCare}, can exist in tons of programming languages.
Due to the language-independence of code smells, code smell detection tools also have the potential to be language-independent.
Emden and Moonen, the builders of the first code smell detection tool, indicated that their detection
approach in Java can be applied to other programming languages \cite{1173068}. A language-independent code smell detection tool can
provide universal detection experience for multi-language software projects. It can also avoid the overhead 
of configuring multiple code smell detection tools with various detection approaches.

In this paper, we present a language-independent code smell detection tool, named TreeNose, to detect 5 types of code smells, Complex Conditional, 
Long Class, Long Method, Long Message Chain, Long Parameter List, across multiple programming languages.
TreeNose implements tree-sitter \cite{treeSitter}, a general parser generator,
to parse the source code of multiple programming languages into AST code structure nodes.
TreeNose can traverse software projects and parse the source code of multiple programming languages 
into AST code structure nodes. After unifying the tree-sitter language-specific nodes,
TreeNose queries the unified nodes to detect targeted code smells with thresholds, which are configured by developers.
TreeNose is designed to be highly extensible, allowing developers to add programming languages without rewriting source code.
  
To evaluate the accuracy of TreeNose,we evaluated TreeNose on 15 open-source projects
implemented in the three chosen programming languages, comparing
its results to those arising from the use of traditional language-specific
code smell detectors. We also conducted evaluations on the characteristics 
of code smells in different programming languages. 

The key contributions of this paper are as follows:

\begin{enumerate}
    \item A language-independent code smell detecter that can detect 5 types of code smells across programming languages.
    \item An evaluation of TreeNose to evaluate its accuracy in detecting code smells in multiple programming languages.
    \item An experiment of TreeNose to reveal the prevalence and distribution of code smells in different programming languages.
\end{enumerate}