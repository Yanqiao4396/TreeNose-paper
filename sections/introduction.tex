\section{Introduction}~\label{sec:introduction}

% Code smells are important
% Manual checking is error-prone
% Automated tools are needed

Code smells are a series of code-design-related concerns that decrease
readability \cite{5741260} and maintainability \cite{6392174,6405287} of
software projects, ultimately limiting the potential for the future maintenance
\cite{Fowler_Beck}. Even though developers acknowledge the negative influence
of code smells on project quality~\cite{developersCare}, manual smell detection
remains a error-prone and resource-consuming process
\cite{DetectingDefectsInObject}. Therefore, code smell detection tools have
been developed to automate this process. Different code smell detection tools
rely on various detection strategies, such as abstract syntax tree (AST)
analysis, machine learning, and static analysis~\cite{ML}. Developers also use
popular, yet language-specific, smell detection tools (e.g., PMD \cite{PMD} and
CheckStyle \cite{CheckStyle}) in the continuous integration setup of
open-source projects like Apache Commons Lang \cite{ApacheCommonsLang} and
Jenkins \cite{Jekins}.

% Multi-language projects are common
% Code smell detection is not commonly language-independent
% Maintaining multiple tools is a cumbersome task

One of the characteristics of modern software projects is the frequent use of
multiple programming languages \cite{723183}. The combination of programming
languages allows developers to mix and match the functionalities and libraries
that are best supported by specific programming language \cite{7476675}.
However, the complexity of multi-language software projects increases the
difficulty of project comprehension and maintenance \cite{7476675,
10.1109/SCAM.2012.11, 7396422}. Along with the complexity, the multi-language
software projects also introduce the pressing challenge of code smell
detection. Most of the existing code smell detection tools are designed to
detect code smells in a single programming language. Multi-language projects,
like Jenkins \cite{Jekins}, implement multiple detection tools in
multi-language software projects, thereby introducing the overhead of
configuring multiple code smell detection tools with various detection
approaches.

While few code smell detection tools are language-independent, most code smells
are. For example, Long Method, one of the most prevalent code smells
\cite{developersCare}, can exist in tons of programming languages. Due to the
language-independence of code smells, code smell detection tools also have the
potential to be language-independent. Van Emden and Moonen, the builders of the
first code smell detection tool, indicated that their detection approach in Java
has the potential to be applied to other programming languages in the future
\cite{1173068}. Abidi~\etal{} built a multi-language design smells (i.e.,
anti-patterns and code smells) detection tool to detect 15
multi-language-specific design smells in system in combinations of Java and
C/C++ \cite{MultiLanguageCodeSmells,Fault-Prone}. A language-independent code
smell detection tool can provide a unified detection experience for
multi-language software projects. It can also avoid the overhead of configuring
multiple code smell detection tools with various detection approaches.

To fill the gap, this paper presents a language-independent code smell detection
tool, named \texttt{TreeNose}, to detect 5 types of code smells, Complex
Conditional, Long Class, Long Method, Long Message Chain, Long Parameter List,
across multiple programming languages. \texttt{TreeNose} implements Treesitter
\cite{treeSitter}, a general parser generator, to parse the source code of
multiple programming languages into the nodes of AST. On top of AST,
\texttt{TreeNose} queries the nodes with detection rules to detect targeted code
smells with thresholds, which are configured by developers. \texttt{TreeNose} is
designed to be highly extensible, allowing developers to add programming
languages without rewriting source code.

To evaluate the accuracy of \texttt{TreeNose}, we evaluated \texttt{TreeNose}
on 9 open-source projects implemented in Java, JavaScript, or Python.
We compared the performance of \texttt{TreeNose} with the combination of 3
language-specific code smell detection tools in a manual annotation study.
We also conducted evaluations on the characteristics of code smells in
different programming languages.

The key contributions of this paper are as follows:

\begin{enumerate}

    \item A language-independent code smell detector that detects 5 types of
        code smells across programming languages.

    \item An evaluation of \texttt{TreeNose} to evaluate its accuracy in
        multiple programming languages.

    \item An experiment to reveal the prevalence and distribution of code smells
        in different programming languages.

\end{enumerate}

Our results show that \texttt{TreeNose} achieved high precision of 92\% and F1
score of 0.94 in detecting selected code smells in target programming languages.
We also found that (1) Complex Conditional counts for 42\% of the total code
smells detected on average, which is the most common code smell in the selected
programming languages. (2) Programming languages have strong tendencies to have
specific code smells, such as JavaScript contains 3 times more Long Method than
other systems. (3) Multi-Language software projects have more evenly distributed
code smells than single-language software projects.
