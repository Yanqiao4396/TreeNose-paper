\section{Introduction}~\label{sec:introduction}

\vspace*{-1em}

% Code smells are important
% Manual checking is error-prone
% Automated tools are needed

Code smells are a series of code-design-related concerns that may decrease
readability \cite{5741260,SANTOS2018450} and maintainability
\cite{6392174,6405287} of software projects, ultimately limiting opportunities
for future maintenance \cite{Fowler_Beck}. Even though developers acknowledge
the negative influence of code smells on project quality~\cite{developersCare},
manual smell detection remains a error-prone and resource-consuming process
\cite{DetectingDefectsInObject}. To automate this process, different code smell
detection tools rely on detection methods like abstract syntax tree (AST)
analysis~\cite{Lenarduzzi2023}, textual analysis~\cite{Palomba}, or machine
learning~\cite{ML}. Developers also use popular, yet language-specific, smell
detection tools (e.g., PMD~\cite{PMD} and CheckStyle~\cite{CheckStyle}) in the
continuous integration setup of projects like Apache Commons
Lang~\cite{ApacheCommonsLang} and Jenkins~\cite{Jekins}.

% Multi-language projects are common
% Code smell detection is not commonly language-independent
% Maintaining multiple tools is a cumbersome task

One of the characteristics of many modern software projects is their use of
multiple programming languages \cite{723183}. The combination of programming
languages allows developers to mix and match the functionalities and libraries
that are best supported by specific programming language \cite{7476675}.
However, the complexity of multi-language software projects increases the
difficulty of project comprehension and maintenance \cite{7476675,
10.1109/SCAM.2012.11, 7396422}. Along with the complexity, these multi-language
software projects also introduce the pressing challenge of code smell detection.
Most of the existing code smell detection tools are designed to detect code
smells in a single programming language. Multi-language projects, like Jenkins
\cite{Jekins}, use more than one detection tool, thereby introducing the
burdensome need to consistently configure multiple smell detection tools.

% There are benefits to having a language-independent smell
% detection tool, but sadly one does really not exist yet!
% This is the pain that we solve in the next paragraph!

While a limited number of code smell detection tools are language-independent,
most code smells are. For example, Long Method, one of the most prevalent code
smells \cite{developersCare}, can exist in many programming languages. Due to
the language-independence of code smells, code smell detection tools also have
the potential to be language-independent. For instance, Van Emden and Moonen,
the creators of an early code smell detection tool, indicated that their
detection approach in Java had the potential to be applied to other programming
languages in the future \cite{1173068}. Furthermore, Abidi~\etal{} built a
multi-language design smells (i.e., anti-patterns and code smells) detection
tool to detect 15 multi-language-specific design smells for programs that
combine Java and C/C++ \cite{MultiLanguageCodeSmells,Fault-Prone}. A truly
language-independent smell detection tool would offer a unified detection
experience for use in both multi-language software projects and across many
single-language projects.

% Describe TreeNose tool and its features,
% making sure to mention that it is language-independent
% and that it is highly extensible due to Treesitter

Addressing this need, this paper presents a language-independent code smell
detection tool, called \texttt{TreeNose}, to detect 5 types of code smells ---
Complex Conditional, Long Class, Long Method, Long Message Chain, Long Parameter
List --- across multiple programming languages. \texttt{TreeNose} leverages
Treesitter \cite{treeSitter}, a general-purpose parser generator, to parse the
source code of multiple programming languages into an AST. Leveraging the
generated AST, \texttt{TreeNose} queries the nodes with detection rules to find,
with configurable thresholds, the targeted code smells. Since it builds on
Treesitter, \texttt{TreeNose} is highly extensible, allowing developers to add
new programming languages without rewriting its source code and to support new
smells with minimal engineering effort.

% Design of the experiments

We experimentally evaluated \texttt{TreeNose} on 9 open-source projects
implemented in Java, JavaScript, or Python, comparing the performance of
\texttt{TreeNose} with the combination of 3 language-specific code smell
detection tools in a manual annotation study. Augmenting the 9 projects with 7
new ones, we also investigated the characteristics of code smells in different
programming languages.
%
% Overview of the experimental results
%
The results showed that \texttt{TreeNose} achieved a precision of 92\% and an F1
score of 0.94 in detecting code smells in the chosen programming languages. We
also found that (i) Complex Conditional accounts for 42\% of the total code
smells detected on average, making it the most common smell in the selected
programming languages; (ii) programming languages have strong tendencies towards
specific smells, such as JavaScript containing 3 times more Long Method smells
than the other languages; (iii) multi-language projects have more evenly
distributed code smells than single-language ones.
%
% Listing of the key contributions
%
The contributions of this paper are:

\begin{enumerate}
    %
    \item A language-independent code smell detector that finds 5 types of code
        smells across programming languages.
        %
    \item An experimental evaluation of \texttt{TreeNose} to study its accuracy
        for multiple programming languages.
        %
    \item An investigation of the prevalence and distribution of code
        smells in different programming languages.
        %
\end{enumerate}
